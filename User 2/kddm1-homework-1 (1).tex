\documentclass[a4paper,10pt]{article}\setlength{\textheight}{10in}\setlength{\textwidth}{6.5in}\setlength{\topmargin}{-0.125in}\setlength{\oddsidemargin}{-.2in}\setlength{\evensidemargin}{-.2in}\setlength{\headsep}{0.2in}\setlength{\footskip}{0pt}\usepackage{amsmath}\usepackage{fancyhdr}\usepackage{enumitem}\usepackage{hyperref}\usepackage{xcolor}\usepackage{graphicx}\usepackage[export]{adjustbox}\usepackage{caption}\usepackage{float}\usepackage{booktabs}\usepackage{makecell}\pagestyle{fancy}

\lhead{Name: \rule{5cm}{0.5pt}}
\chead{M.Number: \rule{2cm}{0.5pt}}
\rhead{KDDM1 VO (INP.31101UF)}
\fancyfoot{}

\begin{document}
\begin{enumerate}[topsep=0mm, partopsep=0mm, leftmargin=*]





%%% Question 1
{\color{blue}
\item\textit{Visual Data Analysis}. Given the dataset ``task1-dataset.ods'' (available in TeachCenter), which comprises a number of features. Provide a number of meaningful visualisations (4 visualisations) that show key properties of the dataset and dependencies. Based on the visualisations provide your interpretation and insights. 
\begin{enumerate}
	\item What pre-processing did your do? (e.g., Did you create new features? Did you normalise the data? Did you filter the dataset? Extended with another dataset?)
	\item What are the most relevant dependencies between the features (selection of the figures)?
	\item Provide a series of meaningful plots that show a specific relationship (dependency) or characteristic of the dataset
	\item Provide a summary of the main insights
\end{enumerate}
}

%%% Your answer here

\textbf{Answer (a)} - Preprocessing steps:
    \begin{itemize}
        \item Handling Missing Values (Removed columns having too many missing values and removed outliers)
        \item Filter the dataset and select necessary features
        \item Rename and Clean Columns (Made column names more readable)
        \item Convert Data types 
        \item Create new features using existence features
        
    \end{itemize}
    
\textbf{Answer (b)} - List of main dependencies:
    \begin{itemize}
        \item The Reported Fatalities by Countries
        \item Fatality rate by Income group
        \item Safety Laws and Fatalities
        \item Speed Limits and Enforcement
    \end{itemize}
    
    
\textbf{Answer (b) and (c)}
% First visualisation
    \begin{figure}[H]
        \centering
        \adjustbox{frame}{
            \includegraphics[width=0.8\textwidth]{top 10 countries by repoted fatalities.png}
        }
        \caption{Top 10 countries by reported fatalities - This bar chart is perfect for study reported fatality rate across countries and seeing how each country stacks up in terms of reported fatality rate, making it easy to spot countries with the highest to lowest reported fatality rates. The chart shows a dependency between the country and its reported fatality rates. }
        \label{fig:plot1}
    \end{figure}
    % Second visualisation
    \begin{figure}[H]
        \centering
        \adjustbox{frame}{
            \includegraphics[width=0.8\textwidth]{avg ft rate by usertype.png}
        }
        \caption{The average fatality rate by User Types}
        \label{fig:plot2}
    \end{figure}
    % Third visualisation
    \begin{figure}[H]
        \centering
        \adjustbox{frame}{
            \includegraphics[width=0.8\textwidth]{vehicle vs rate by income.png}
        }
        \caption{The Vehicles per 100,000 Population vs Fatality Rate by Income group}
        \label{fig:plot3}
    \end{figure}
    % Fourth visualisation
    \begin{figure}[H]
        \centering
        \adjustbox{frame}{
            \includegraphics[width=0.8\textwidth]{key laws vs fa rate.png}
        }
        \caption{Key Law presence vs estimated fatality Rate}
        \label{fig:plot4}
    \end{figure}

\textbf{Answer (d)} - Short summary of the main insights (with references to the corresponding image)
    \begin{itemize}
        \item Finding \#1, as illustrated in Figure~\ref{fig:plot1}  highlights the countries with the highest number of reported fatalities. India stands out with the highest fatality count, reporting 153,972 deaths. China follows with 62,218 fatalities, and the United States is third with 42,939 fatalities. Beyond these three, the numbers gradually decline, with Brazil, Indonesia, Thailand, the Islamic Republic of Iran, the Russian Federation, Mexico, and South Africa completing the top 10 list.
        
        These figures reflect the significant burden of road traffic fatalities in these countries, underscoring the urgent need for stronger road safety measures and interventions to save lives.

        \item Finding \#2, as illustrated in Figure~\ref{fig:plot2}, presents the average distribution of fatalities across different vehicle user types. The highest proportion of fatalities is seen among occupants of light powered vehicles. This is followed by pedestrians, who represent the second most affected group. Two-wheel powered vehicle users, such as motorcyclists, come next, while cyclists account for the smallest share of fatalities.
        

        \item Finding \#3, as illustrated in Figure~\ref{fig:plot3}, illustrates the relationship between the number of registered vehicles per 100,000 population and the fatality rate per 100,000 population, colored by income group.

The data reveals that high-income countries have the highest number of registered vehicles, yet they experience relatively low fatality rates. In contrast, low-income countries have far fewer registered vehicles but suffer from significantly higher fatality rates.

Among the middle-income groups, upper-middle-income countries show a higher number of registered vehicles compared to lower-middle-income countries. However, lower-middle-income countries exhibit a higher fatality rate than their upper-middle-income counterparts.

These findings highlight the disparities in road safety outcomes across different income groups and emphasize the need for improved road safety infrastructure and interventions, particularly in lower-income settings.
        
        \item Finding \#4, as illustrated in Figure~\ref{fig:plot4}, the boxplot reveals the fatality rate per 100,000 population is analyzed in relation to the presence of key road safety laws, such as national motorcycle helmet laws, national seat-belt laws, national child restraint laws, and national speed limit laws. The results show that countries with these laws in place generally report a narrower range of fatality rates and a lower median fatality rate compared to those without such regulations. These findings highlight the significant impact that comprehensive road safety laws have in mitigating traffic-related fatalities.
    \end{itemize}






\clearpage
{\color{blue}
\newpage\item\textit{Correlation}. Given a dataset, which consists of 1,000 variables (hint: most of them are just random), the goal is to find the relationships between variables, i.e., which and how do the variables relate to each other; what are the dependencies.
The dataset ``task2-dataset.csv'' can be downloaded from TeachCenter.
\begin{enumerate}
	\item Which methods did you apply to find the relationships, and why?
	\item Which relationships did you find and how do you characterise the relationships (e.g., variable ``Lurkowl (Strix umbra \#1068)'' to ``Frosthawk (Accipiter glacies \#1064)'' is linear with correlation found via method X of 0.9)?
	\item Which causal relationships between the variables can you find (e.g., variable ``Rattlepuff (Lynx rattleus \#1067)'' causes ``Slingshark (Carcharodon slingus \#1068)'')?
\end{enumerate}
}

%%% Your answer here
\textbf{Answer (a)} - Method and motivation:
    \begin{enumerate}
        \item Pearson Correlation Coefficient Method - Used because it provides a quantifiable way to assess how strongly two variables are related and in what direction (positive or negative).
        \item Regression Analysis Method - Used to examine the relationship between a dependent (or response) variable and one or more independent (or predictor) variables.
        \item Scatter Plot Method - Helps to visualize correlation and indicate the nature of relationships.
        
    \end{enumerate}

\textbf{Answer (b)}
\begin{center}
\begin{tabular}{lllll}
\toprule
\textbf{Variable 1} & \textbf{Variable 2} & \textbf{Type of dependency} & \textbf{Method} & \textbf{Value} \\ \midrule
\makecell{Lurkowl \\ (Strix umbra #1068)} & \makecell{Trunkasaurus \\ (Trunkasaurus ancientus #304)} & Linear & Correlation & -0.932852 \\
\makecell{Chirpsnail \\ (Cornu chirpitus #556)} & \makecell{Crunchbeetle \\ (Coleoptera crunchus #1081)} & Linear & Correlation & 0.992159 \\
\makecell{Slingshark \\ (Carcharodon slingus #894)} & \makecell{Squeakfluff \\ (Sorex squeakus #1070)} & Linear & Correlation & -0.998983 \\
\makecell{Fluffernox \\ (Leontodon fluffernus #778)} & \makecell{Driftwolf \\ (Canis fluctus #1065)} & Linear & Correlation & 0.950261 \\
\makecell{Skydash Falcon \\ (Falco aeris #490)} & \makecell{Shimmercat \\ (Felis lucere #323)} & Linear & Correlation & 0.934820 \\
\bottomrule
\end{tabular}
\end{center}


\textbf{Answer (c)}
    \begin{itemize}
        \item "Chirpsnail (Cornu chirpitus #556)" causes "Crunchbeetle (Coleoptera crunchus #1081)"\\ (Strong positive linear relationship - When "Chirpsnail (Cornu chirpitus #556)" increases, "Crunchbeetle (Coleoptera crunchus #1081) also increases. )
        \item "Lurkowl (Strix umbra #1068)" causes "Trunkasaurus (Trunkasaurus ancientus #304)"\\ (Strong negative linear relationship - When "Lurkowl (Strix umbra #1068)" increases, "Trunkasaurus (Trunkasaurus ancientus #304) is decreases.)
        \item "Fluffernox (Leontodon fluffernus 778)" causes "Driftwolf (Canis fluctus 1065)"\\ (Strong positive linear relationship - When "Fluffernox (Leontodon fluffernus 778)" increases, "Driftwolf (Canis fluctus 1065) also increases. )
        
    \end{itemize}


{\color{blue}
\clearpage\item \textit{Outliers/Anomalies}. Given two types of anomalies: (1) anomalies are defined to be \textbf{datapoints} in low density regions, (2) anomalies are \textbf{regions} of low density.
\begin{enumerate}
\item For both anomalies please create/draw a dataset of 2 features (x and y axis), with 3 anomalies and many normal data points (the normal datapoints should be marked, e.g., green colour)
\item Name the algorithms or describe the algorithmic way of how to identify this anomalous behaviour (you may also describe any necessary preprocessing)
\item Name the assumptions made by your algorithms
\end{enumerate}}




\textbf{Answer (a)} - Draw two datasets % Update the images to reflect a scatterplot of a dataset
    \begin{center}
    \includegraphics[width=.4\textwidth]{img1.png}
        \hspace{2cm}
    \includegraphics[width=.4\textwidth]{img2.png}
    \end{center}


\textbf{Answer (b)} - Describe the algorithms 
\begin{description}
	\item[Dataset 1] ``Anomalies as Individual Points Algorithm'' starts by creating normal data points clustered around the center of the grid at (50, 50), using a Gaussian distribution. Then, anomalies are introduced in low-density regions at the grid's corners, such as (10, 10), (90, 10), and (10, 90). After generating the data, the normal points are plotted in green, and the anomalies are marked in red "X" on a scatter plot. A grid with tick marks at intervals of 10 is added for clarity, with axis limits from 0 to 100.
	\item[Dataset 2] "Anomalies in Isolated Regions Algorithm" was used normal data points are clustered around (50, 50), while anomalies are placed in isolated regions far from the main cluster, such as (0, 0), (100, 0), and (0, 100). The normal data is plotted in green, and the anomalies are marked with orange "X". Similar to Dataset 1, the plot includes a grid with tick marks every 10 units, and the axis limits are set from 0 to 100.
\end{description}


\textbf{Answer (c)} - Describe the main assumptions \\
\begin{center}
\begin{tabular}{ll}
\toprule
\textbf{Algorithm} & \textbf{Assumption} \\ \midrule
Anomalies as Individual Points & Normal data are clustered \\
Anomalies as Individual Points & Anomalies are located at a significant distance from the normal data clusters. \\ 
Anomalies as Individual Points & Gaussian distribution approximates the normal behavior \\
Anomalies in Isolated Regions & Anomalies are from the low-density region \\
Anomalies in Isolated Regions & Both x and y dimensions are considered equally important \\ 
\bottomrule
\end{tabular}
\end{center}


%%% Your answer here



{\color{blue}
\clearpage\item\textit{Missing Values}. The dataset ``task4-dataset.csv'' (available on TeachCenter) contains a number of missing values. Try to reconstruct why the missing values are missing? What could be an explanation?
\begin{enumerate}
	\item What are the dependencies in the dataset?
	\item What could be reasons for the missingness?
	\item What strategies are applicable for the features to deal with the missing values?
	\item For each feature provide an estimate of the arithmetic mean (before and after applying the strategies to deal with missing values)?
\end{enumerate}
}

%%% Your answer here
\textbf{Answer (a)} - Describe the dependencies in the dataset
\begin{center}
\begin{tabular}{lll}
\toprule
\textbf{X} & \textbf{Y} & \textbf{Type of dependency} \\ \midrule
gender & height & Association \\
age & semester & Positive correlation \\
books per year & english skills & Weak positive correlation \\
\bottomrule
\end{tabular}
\end{center}

\textbf{Answer (b)} - Describe the reason for missingness
\begin{center}
\begin{tabular}{ll}
\toprule
\textbf{Variable} & \textbf{Reason} \\ \midrule
height & Measurement not taken or Data entry error \\
likes pinapple on pizza  & Forgot to answer, or skipped \\
english skills & Missing not at random or preferred not to disclose \\
semester & Missing at random \\
\bottomrule
\end{tabular}
\end{center}

\textbf{Answer (c)} - Describe the strategies for dealing with missing values
\begin{center}
\begin{tabular}{ll}
\toprule
\textbf{Variable} & \textbf{Strategy} \\ \midrule
height & Fill missing values with the median by grouping the data based on gender \\
likes pinapple on pizza  & Created new category as unknown nulls \\
english skills & Null values were filled with median \\
semester & Removed semester data greater than 40 as outliers \\

\bottomrule
\end{tabular}
\end{center}

\textbf{Answer (d)} - Arithmetic mean of original dataset (with the missing values), and the one after applying the strategies
\begin{center}
\begin{tabular}{lll}
\toprule
\textbf{Variable} & \textbf{Before Strategy} & \textbf{After Strategy} \\ \midrule
height & 172.161446 & 172.764921 \\
semester & 26.1560 & 10.343484 \\
english skills & 87.009721 & 87.133983 \\

\bottomrule
\end{tabular}
\end{center}

\end{enumerate}
\end{document}
