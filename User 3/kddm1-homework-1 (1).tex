\documentclass[a4paper,10pt]{article}\setlength{\textheight}{10in}\setlength{\textwidth}{6.5in}\setlength{\topmargin}{-0.125in}\setlength{\oddsidemargin}{-.2in}\setlength{\evensidemargin}{-.2in}\setlength{\headsep}{0.2in}\setlength{\footskip}{0pt}\usepackage{amsmath}\usepackage{fancyhdr}\usepackage{enumitem}\usepackage{hyperref}\usepackage{xcolor}\usepackage{graphicx}\usepackage[export]{adjustbox}\usepackage{caption}\usepackage{float}\usepackage{booktabs}\usepackage{makecell}\pagestyle{fancy}

\lhead{Name: \rule{5cm}{0.5pt}}
\chead{M.Number: \rule{2cm}{0.5pt}}
\rhead{KDDM1 VO (INP.31101UF)}
\fancyfoot{}

\begin{document}
\begin{enumerate}[topsep=0mm, partopsep=0mm, leftmargin=*]





%%% Question 1
{\color{blue}
\item\textit{Visual Data Analysis}. Given the dataset ``task1-dataset.ods'' (available in TeachCenter), which comprises a number of features. Provide a number of meaningful visualisations (4 visualisations) that show key properties of the dataset and dependencies. Based on the visualisations provide your interpretation and insights. 
\begin{enumerate}
	\item What pre-processing did your do? (e.g., Did you create new features? Did you normalise the data? Did you filter the dataset? Extended with another dataset?)
	\item What are the most relevant dependencies between the features (selection of the figures)?
	\item Provide a series of meaningful plots that show a specific relationship (dependency) or characteristic of the dataset
	\item Provide a summary of the main insights
\end{enumerate}
}

%%% Your answer here

\textbf{Answer (a)} - Preprocessing steps:
    \begin{itemize}
        \item Convert Data types
        \item Rename and Clean Columns (Made column names more readable)
        \item Filter the dataset and select the necessary features
        \item Handling Missing Values (Removed columns having too many missing values and removed outliers) 
        \item Create new features using existing features
        
    \end{itemize}
    
\textbf{Answer (b)} - List of main dependencies:
    \begin{itemize}
        \item The reported fatalities by countries
        \item The year by the fatality reduction target by countries
        \item Safety laws and fatalities
        \item The participation in the GRSSR of countries 
    \end{itemize}
    
    
\textbf{Answer (b) and (c)}
% First visualisation
    \begin{figure}[H]
        \centering
        \adjustbox{frame}{
            \includegraphics[width=0.8\textwidth]{Top 10 countries by reported fatalities.png}
        }
        \caption{Top 10 countries by reported fatalities -This bar chart is ideal for analyzing the reported fatality rates across different countries, allowing for an easy comparison of how each country ranks. It clearly highlights the variation in reported fatality rates, showing the relationship between each country and its corresponding fatality rate.}
        \label{fig:plot1}
    \end{figure}
    % Second visualisation
    \begin{figure}[H]
        \centering
        \adjustbox{frame}{
            \includegraphics[width=0.8\textwidth]{year by fatality reduction target.png}
        }
        \caption{The top 10 countries of the year by fatality reduction target}
        \label{fig:plot2}
    \end{figure}
    % Third visualisation
    \begin{figure}[H]
        \centering
        \adjustbox{frame}{
            \includegraphics[width=0.8\textwidth]{Key law precense vs road fatality rate.png}
        }
        \caption{The Key laws presence vs road fatality rate}
        \label{fig:plot3}
    \end{figure}
    % Fourth visualisation
    \begin{figure}[H]
        \centering
        \adjustbox{frame}{
            \includegraphics[width=0.8\textwidth]{country participation.png}
        }
        \caption{The Participation of countries for Global Road Safety Status Reports}
        \label{fig:plot4}
    \end{figure}

\textbf{Answer (d)} - Short summary of the main insights (with references to the corresponding image)
    \begin{itemize}
        \item Finding \#1, as illustrated in Figure~\ref{fig:plot1} brings into focus the countries with the highest number of reported road traffic fatalities. India leads by a significant margin, with 153,972 deaths, followed by China with 62,218, and the United States with 42,939 fatalities. After these three, the numbers steadily decrease, with Brazil, Indonesia, Thailand, the Islamic Republic of Iran, the Russian Federation, Mexico, and South Africa rounding out the top 10.

These figures highlight the heavy toll that road traffic accidents continue to take in these countries. They serve as a powerful reminder of the urgent need for stronger road safety policies and interventions to protect lives on our roads.

        \item Finding \#2, as illustrated in Figure~\ref{fig:plot2}, presents the year-by-fatality reduction targets for the top 10 countries with the most ambitious road safety goals. The Netherlands leads with the longest-term target set for 2050, reflecting a strong, long-term commitment to eliminating road fatalities. Following closely are Indonesia and the United States, both aiming for 2040. Other countries, including Iceland, Colombia, Algeria, Australia, Austria, Bahrain, and Bangladesh, have also established clear timelines for reducing road deaths. This comparison highlights varying levels of ambition and urgency among nations, emphasizing the global recognition of road safety as a public health priority and the importance of setting measurable, time-bound targets to drive progress.
        

        \item Finding \#3, as illustrated in Figure~\ref{fig:plot3}, presents the fatality rate per 100,000 population in relation to the presence of key road safety laws, including national motorcycle helmet laws, national seat-belt laws, national child restraint laws, and national laws setting speed limits. The analysis reveals that countries where these laws are in place tend to have a lower range of fatality rates, as well as a lower median fatality rate, compared to countries without such legislation. These findings underscore the critical role that comprehensive road safety laws play in reducing traffic-related fatalities.
        
        \item Finding \#4, as illustrated in Figure~\ref{fig:plot4},  illustrates the participation of countries in the GRSSR (Global Road Safety Status Report) commitments across the years 2009, 2013, 2015, and 2018. The data show that the level of participation remained relatively consistent over these years, with no substantial changes observed. This steady engagement suggests a positive commitment toward ongoing efforts to reduce road traffic fatalities.
    \end{itemize}






\clearpage
{\color{blue}
\newpage\item\textit{Correlation}. Given a dataset, which consists of 1,000 variables (hint: most of them are just random), the goal is to find the relationships between variables, i.e., which and how do the variables relate to each other; what are the dependencies.
The dataset ``task2-dataset.csv'' can be downloaded from TeachCenter.
\begin{enumerate}
	\item Which methods did you apply to find the relationships, and why?
	\item Which relationships did you find and how do you characterise the relationships (e.g., variable ``Lurkowl (Strix umbra \#1068)'' to ``Frosthawk (Accipiter glacies \#1064)'' is linear with correlation found via method X of 0.9)?
	\item Which causal relationships between the variables can you find (e.g., variable ``Rattlepuff (Lynx rattleus \#1067)'' causes ``Slingshark (Carcharodon slingus \#1068)'')?
\end{enumerate}
}

%%% Your answer here
\textbf{Answer (a)} - Method and motivation:
    \begin{enumerate}
        \item Pearson Correlation Coefficient Method - Used because it provides a quantifiable way to assess how strongly two variables are related and in what direction (positive or negative).
        \item Scatter Plot Method - Helps to visualize correlation and indicate the nature of relationships.
        \item Regression Analysis Method - Used to examine the relationship between a dependent (or response) variable and one or more independent (or predictor) variables.
    \end{enumerate}

\textbf{Answer (b)}
\begin{center}
\begin{tabular}{lllll}
\toprule
\textbf{Variable 1} & \textbf{Variable 2} & \textbf{Type of dependency} & \textbf{Method} & \textbf{Value} \\ \midrule
\makecell{Danglefawn \\ (Cervus dangleus #1067)} & \makecell{Puffpounce \\ (Felis floccus #196)} & Linear & Correlation & 0.999867 \\
\makecell{Chirpsnail \\ (Cornu chirpitus #556)} & \makecell{Crunchbeetle \\ (Coleoptera crunchus #1081)} & Linear & Correlation & 0.992159 \\
\makecell{Slingshark \\ (Carcharodon slingus #894)} & \makecell{Squeakfluff \\ (Sorex squeakus #1070)} & Linear & Correlation & -0.998983 \\
\makecell{Sparklequail \\ (Coturnix fulgor #1068)} & \makecell{Scruffpaws \\ (Felis scruffus #1061)} & Linear & Correlation & 0.956447 \\
\makecell{Vinepaw \\ (Felis liana #1073)} & \makecell{Driftwolf \\ (Canis fluctus #542)} & Linear & Correlation & -0.942152 \\
\bottomrule
\end{tabular}
\end{center}


\textbf{Answer (c)}
    \begin{itemize}
        \item "Danglefawn (Cervus dangleus #1067)" causes "Puffpounce (Felis floccus #196)"\\ (Strong positive linear relationship - When "Chirpsnail Danglefawn (Cervus dangleus #1067)" increases, "Puffpounce (Felis floccus #196) also increases. )
        \item "Sparklequail (Coturnix fulgor #1068)" causes "Scruffpaws (Felis scruffus #1061)"\\ (Strong positive linear relationship - When "Sparklequail (Coturnix fulgor #1068)" increases, "Scruffpaws (Felis scruffus #1061) also increases. )
        \item "Vinepaw (Felis liana #1073)" causes "Driftwolf (Canis fluctus #542)"\\ (Strong negative linear relationship - When "Vinepaw (Felis liana #1073)" increases, "Driftwolf (Canis fluctus #542) is decreases.)
    \end{itemize}


{\color{blue}
\clearpage\item \textit{Outliers/Anomalies}. Given two types of anomalies: (1) anomalies are defined to be \textbf{datapoints} in low density regions, (2) anomalies are \textbf{regions} of low density.
\begin{enumerate}
\item For both anomalies please create/draw a dataset of 2 features (x and y axis), with 3 anomalies and many normal data points (the normal datapoints should be marked, e.g., green colour)
\item Name the algorithms or describe the algorithmic way of how to identify this anomalous behaviour (you may also describe any necessary preprocessing)
\item Name the assumptions made by your algorithms
\end{enumerate}}




\textbf{Answer (a)} - Draw two datasets % Update the images to reflect a scatterplot of a dataset
    \begin{center}
    \includegraphics[width=.4\textwidth]{img1.png}
        \hspace{2cm}
    \includegraphics[width=.4\textwidth]{img2.png}
    \end{center}


\textbf{Answer (b)} - Describe the algorithms 
\begin{description}
	\item[Dataset 1] ``Anomalies as Data Points in Low-Density Regions'' : A fixed random seed is set for reproducibility. Ninety-seven normal data points are generated from a multivariate normal distribution centered at [50, 50] with a small covariance. Anomalies are added at sparse locations: [10, 9.9], [99, 98.5],  and [8, 78]. The normal data and anomalies are combined into a single dataset, with labels assigned as 0 for normal points and 1 for anomalies.
	\item[Dataset 2] "Anomalies as Regions of Low Density (Hole in the Center)": A random seed is set, and ninety-seven normal data points are generated in a ring pattern around [50, 50] with random radii between 40 and 50. Anomalies are placed near the center of the ring at [48, 52], [36, 40], and [50, 52]. The normal and anomaly points are merged into a dataset, with labels assigned as 0 for normal points and 1 for anomalies.
\end{description}


\textbf{Answer (c)} - Describe the main assumptions \\
\begin{center}
\begin{tabular}{ll}
\toprule
\textbf{Algorithm} & \textbf{Assumption} \\ \midrule
\makecell{Anomalies as Data Points \\ in Low-Density Regions} & Normal data are clustered \\
\makecell{Anomalies as Data Points \\ in Low-Density Regions} & \makecell{These outliers are positioned far from areas \\ where normal data is concentrated.} \\ 
\makecell {Anomalies as Data Points \\ in Low-Density Regions} & \makecell {The labels are assumed to be binary, with 0 for normal points \\ and 1 for anomaly points} \\
\makecell {Anomalies as Regions of Low Density \\ (Hole in the Center)} & \makecell{The data follows this ring-shaped distribution and the radius of the ring \\ is randomly determined} \\
\makecell {Anomalies as Regions of Low Density \\ (Hole in the Center)} & \makecell {Anomalies are assumed to occur in a small, distinct area within the center \\ of the ring} \\ 
\makecell{Anomalies as Regions of Low Density \\ (Hole in the Center)} & \makecell {There is no overlap between the normal data points and the anomalies} \\
\bottomrule
\end{tabular}
\end{center}


%%% Your answer here



{\color{blue}
\clearpage\item\textit{Missing Values}. The dataset ``task4-dataset.csv'' (available on TeachCenter) contains a number of missing values. Try to reconstruct why the missing values are missing? What could be an explanation?
\begin{enumerate}
	\item What are the dependencies in the dataset?
	\item What could be reasons for the missingness?
	\item What strategies are applicable for the features to deal with the missing values?
	\item For each feature provide an estimate of the arithmetic mean (before and after applying the strategies to deal with missing values)?
\end{enumerate}
}

%%% Your answer here
\textbf{Answer (a)} - Describe the dependencies in the dataset
\begin{center}
\begin{tabular}{lll}
\toprule
\textbf{X} & \textbf{Y} & \textbf{Type of dependency} \\ \midrule
age & semester & Positive correlation \\
books per year & english skills & Weak positive correlation \\
gender & height & Association \\
gender & likes chocolate & Association \\
\bottomrule
\end{tabular}
\end{center}

\textbf{Answer (b)} - Describe the reason for missingness
\begin{center}
\begin{tabular}{ll}
\toprule
\textbf{Variable} & \textbf{Reason} \\ \midrule
height & Measurements not taken or Data entry error \\
likes pinapple on pizza  & Omitted or missed answering \\
english skills & Data missing randomly or Chose not to disclose  \\
semester & Missing at random \\
\bottomrule
\end{tabular}
\end{center}

\textbf{Answer (c)} - Describe the strategies for dealing with missing values
\begin{center}
\begin{tabular}{ll}
\toprule
\textbf{Variable} & \textbf{Strategy} \\ \midrule
height & Fill missing values with the median by grouping the data based on gender \\
semester & Eliminated data points from the semester where values surpassed 40 \\
likes pinapple on pizza  & New category as unknown was created \\
english skills & Filled with median \\

\bottomrule
\end{tabular}
\end{center}

\textbf{Answer (d)} - Arithmetic mean of original dataset (with the missing values), and the one after applying the strategies
\begin{center}
\begin{tabular}{lll}
\toprule
\textbf{Variable} & \textbf{Before Strategy} & \textbf{After Strategy} \\ \midrule
height & 172.161446 & 172.764921 \\
english skills & 87.009721 & 87.133983 \\
semester & 26.1560 & 10.343484 \\
\bottomrule
\end{tabular}
\end{center}

\end{enumerate}
\end{document}
